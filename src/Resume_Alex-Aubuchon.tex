\documentclass[margin,line,11pt,letterpaper]{res}

    \usepackage[colorlinks=false]{hyperref}
    \hypersetup{
        colorlinks=false,
        linkcolor=blue,
        filecolor=magenta,      
        urlcolor=cyan,
    }

    % Side section header styling
    \let\oldsection\section
    \renewcommand{\section}[1]{\vspace{-2mm}\oldsection{\small\sc {#1}}}
    
\begin{document}
\newsectionwidth{1in}

\name{\makebox[\resumewidth]{{\LARGE Alexander Sedgwick Aubuchon} \hfill}}

\address{114 Hemenway St Apt 3, Boston, MA 02115}

\begin{resume}
    \vspace{-7mm}
    \section{Contact Information}
    Phone: +1 978 894-6108\hfill GitHub: \url{https://github.com/a-lxe}\\
    Mail: \href{mailto:alex@alxe.me}{alex@alxe.me} \hfill LinkedIn: \url{https://linkedin.com/in/a-lxe}
    
    
    \section{Education}
    
    \begin{format}
        \employer{l}\location{r}\\
        \title{l}\dates{r}\\
        \body\\
    \end{format}
    
    \employer{\textbf{Northeastern University}}
    \title{\textit{BS in Computer Science}}
    \dates{\textit{September 2015 - Expected Dec 2019}}
    \location{Boston, MA}
    \begin{position}
        \vspace{-3mm}\\
        Coursework: Advanced Algorithms, Theory of Comp., NLP, Quantum Mechanics, Electronics\\
        Extracurriculars: Tutor for Discrete Structures (2017), Upperclassman Tutor (2016) \\
        GPA: 3.8/4.0
    \end{position}
    
    \section{Work Experience}
    
    \employer{\textbf{CERN, Compact Muon Solenoid}}
    \title{\textit{Software Engineer and Data Analyst}}
    \dates{\textit{January 2018 - August 2018}}
    \location{Geneva, Switzerland}
    \begin{position}
        \vspace{-3mm}\\
        Worked in the Endcap Muon Track Finder (EMTF) group and closely with the Cathode Strip Chamber (CSC) group:
        \begin{list2}
            \item Analyzed data from the EMTF and presented on predictions of detector performance after the HL-LHC upgrade where luminosity and event rate will increase massively.
            \item Designed and deployed the AutoDQM project - a platform for performing statistical and ML-based tests comparing the data quality monitoring plots between two accelerator runs.
        \end{list2}
    \end{position}

    \employer{\textbf{Draper Laboratories}}
    \title{\textit{Software Engineer, Machine Intelligence Group}}
    \dates{\textit{January 2017 - August 2017}}
    \location{Cambridge, MA}
    \begin{position}
        \vspace{-3mm}\\
        Involved in projects focusing on prosthetics, electronics, and machine learning:
        \begin{list2}
            \item Implemented the software interface to a neuro-stimulation device.
            Designed the next-gen device using wirelessly controlled implants collaborating closely with the hardware/firmware teams. 
            Proposed a schema for communicating the low-latency stimulation control over Bluetooth.
            \item Performed research into the novel uses of RF sensors on the emissions of electronic devices.
            \item Integrated new features into an SVM machine learning model designed to use human biometrics to non-invasively detect malintent.
        \end{list2}
    \end{position}
    
    \employer{\textbf{MIT Media Lab}}
    \title{\textit{Assistant Researcher, Changing Places Lab}}
    \dates{\textit{July 2016 - January 2018}}
    \location{Cambridge, MA}
    \begin{position}
        \vspace{-3mm}\\
        Worked on the CityMatrix project culminating in a \href{https://bit.ly/2xqEsnW}{paper} accepted to Acadia 2018:
        \begin{list2}
            \item Researched ML models for quickly predicting the traffic and solar potential of a model city.
            \item Implemented searching over possible changes to the city for an AI assistant to offer suggestions to optimize city performance metrics.
            \item Developed a Unity/C\# application for real-time visualization of the model city.
        \end{list2}
    \end{position}
    
    \section{Projects + Competitions}
    
    \begin{format}
        \title{l}\dates{r}\\
        \body\\
    \end{format}
    
    \title{\textbf{HackMIT: PacTravel}}
    \dates{\textit{September 2017}}
    \begin{position}
        \vspace{-3mm} \\
        Led the development of a webapp in Typescript where users play PacMan on Google Maps while the places they visit are logged into a data-rich itinerary for use afterwards in real life.
    \end{position}
    
    \title{\textbf{HackTheMachine - 2nd Place}}
    \dates{\textit{September 2017}}
    \begin{position}
        \vspace{-3mm} \\
        Used correlations and PCA to improve understanding of highly dimensional Navy ship sensor data.
    \end{position}
    
    \title{\textbf{Google IO Firebass}}
    \dates{\textit{July 2016, May 2017}}
    \begin{position}
        \vspace{-3mm}\\
        Won a global ARG to chase down the 'Firebass' by solving puzzles as it fled through the web.
    \end{position}
    
    \section{Computer Skills}
    \renewcommand{\linespread}{5}
    \textbf{Languages}:
    \textit{Extensive-} Python; Typescript;
    \textit{Versed-} C\#; C/C++; Racket; Java; Rust \\
    \textbf{WebDev}:
    REST; Node.js/npm; React; webpack; HTML/CSS \\
    \textbf{Tools}:
    Unity; Mathematica; Photoshop; Blender (CAD); ROOT \\
    \textbf{Misc}:
    Git; (Arch)Linux; Vim; \LaTeX
     
    \section{Interests}
    Gaming; Rock Climbing; Computer Design; Electronics
\end{resume}
\end{document}
