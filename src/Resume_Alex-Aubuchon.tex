\documentclass[margin,line,11pt,letterpaper]{res}

    \usepackage[colorlinks=false]{hyperref}
    \hypersetup{
        colorlinks=false,
        linkcolor=blue,
        filecolor=magenta,      
        urlcolor=cyan,
    }

    % Side section header styling
    \let\oldsection\section
    \renewcommand{\section}[1]{\vspace{-2mm}\oldsection{\small\sc {#1}}}
    
\begin{document}
\newsectionwidth{1in}

\name{\makebox[\resumewidth]{{\LARGE Alex Aubuchon}\hfill \textit{\textmd{Be a person others look for to solve problems, and make waves.}}}}

\address{114 Hemenway St Apt 3, Boston, MA 02115}

\begin{resume}
    \vspace{-7mm}
    \section{Contact Information}
    Phone: (978) 894-6108\hfill GitHub: \url{https://github.com/a-lxe}\\
    Mail: \href{mailto:alex@alxe.me}{alex@alxe.me} \hfill Linkedin: \url{https://linkedin.com/in/a-lxe}
    
    
    \section{Education}
    
    \begin{format}
        \employer{l}\location{r}\\
        \title{l}\dates{r}\\
        \body\\
    \end{format}
    
    \employer{\textbf{Northeastern University}}
    \title{\textit{BS in Computer Science and Physics}}
    \dates{\textit{September 2015 - Expected May 2019}}
    \location{Boston, MA}
    \begin{position}
        \vspace{-3mm}\\
        Coursework: Advanced Algorithms, Theory of Comp, NLP, Quantum Mechanics, Electronics\\
        GPA: 3.8/4.0
    \end{position}
    
    \section{Work Experience}

    \employer{\textbf{CERN}}
    \title{\textit{Research Assistant, CMS Experiment}}
    \dates{\textit{January 2018 - August 2018}}
    \location{Geneva, CH}
    \begin{position}
        Performed research and software development for the Endcap Muon Track Finder (EMTF):
        \begin{list2}
            \item Developed the AutoDQM system -- a platform to accelerate data quality monitoring by automatically flagging anomalous system metrics via comparison.
            \item Analyzed detector performance of the Endcap Muon TrackFinder and extrapolated results to  account for the High-Luminosity Large Hadron Collider upgrade.
        \end{list2}
    \end{position}
    
    \employer{\textbf{Draper Laboratories}}
    \title{\textit{Software Engineer, Machine Intelligence Group}}
    \dates{\textit{January 2017 - August 2017}}
    \location{Cambridge, MA}
    \begin{position}
        \vspace{-3mm}\\
        Involved in three projects across multiple disciplines including prosthetics, electronics, and ML:
        \begin{list2}
            \item Implemented the software interface in C\# to a neuro-stimulation device as part of the DARPA HAPTIX program.
            Designed the next generation of the device using wirelessly controlled implants by working closely with the hardware and firmware teams. 
            \item Performed research + experimentation into the novel uses of RF sensors on the spurious emissions of electronic devices.
            Used Arduino/C++ for hardware and Python for testing.
            \item Integrated new features into a SVM to use human biometrics to non-invasively detect malintent.
        \end{list2}
    \end{position}
    
    \employer{\textbf{MIT Media Lab}}
    \title{\textit{Assistant Researcher, Changing Places Lab}}
    \dates{\textit{July 2016 - Present}}
    \location{Cambridge, MA}
    \begin{position}
        \vspace{-3mm}\\
        Developed the software/ML oriented parts of the CityMatrix project:
        \begin{list2}
            \item Designed ML models for quickly predicting the traffic and solar power potential of a model city.
            \item Implemented searching over possible changes to the city for a mock intelligence to offer suggestions to the user to optimize these metrics (and others).
            \item Developed a Unity/C\# application for real-time visualization of the model city.
        \end{list2}
    \end{position}
    
    \section{Projects + Competitions}
    
    \begin{format}
        \title{l}\dates{r}\\
        \body\\
    \end{format}
    
    \title{\textbf{HackMIT: PacTravel}}
    \dates{\textit{September 2017}}
    \begin{position}
        \vspace{-3mm} \\
        Led the development of a webapp in Typescript where users play PacMan on Google Maps while the places they visit are logged into a data-rich itinerary for use afterwards in real life.
    \end{position}
    
    \title{\textbf{HackTheMachine - 2nd Place}}
    \dates{\textit{September 2017}}
    \begin{position}
        \vspace{-3mm} \\
        Won second place in a data science hackathon to use Navy ship sensors to improve readiness. 
        Used correlations and PCA in Python to improve human readability of the high dimensional stream.
    \end{position}
    
    \title{\textbf{Google IO Firebass}}
    \dates{\textit{July 2016, May 2017}}
    \begin{position}
        \vspace{-3mm}\\
        Won a global ARG competition to be the first to chase down the Firebase Firebass by solving puzzles as it fled through the web.
        Received tickets to Google IO 2017, attended May.
    \end{position}
    
    \section{Computer Skills}
    \renewcommand{\linespread}{5}
    \textbf{Languages}:
    \textit{Lots-} C\#; Python; Java;
    \textit{Some-} C/C++; Racket; Matlab; Verilog;
    \textit{Little-} Julia \\
    \textbf{WebDev}:
    REST/Websocket APIs; TypeScript; AngularJS; NodeJS; HTML/CSS; Three.js \\
    \textbf{Tools}:
    Unity; Mathematica; Photoshop; Blender (CAD); Microsoft Office \\
    \textbf{Misc}:
    Git; (Arch)Linux; Vim
     
    
    \section{Interests}
    Gaming; Rock Climbing; Transhumanism; CS/Physics Theory; Computer Design; Electronics
\end{resume}
\end{document}
